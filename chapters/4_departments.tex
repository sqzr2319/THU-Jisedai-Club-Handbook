    \fontsize{23pt}{24pt}\selectfont
    \textbf{\textcolor{truepurple}{次世代cosplay舞台剧部}}\\
\vspace{0.7em}
  \adjustbox{valign=t}{
    \begin{minipage}[t]{0.25\textwidth}
      \vspace{-0.5em}
      \raisebox{-\height}{
      \includegraphics[width=\linewidth]{部酱/cosplay舞台剧.png}}
      
      \picbox{\small ~~\ding{115} ~ cosplay舞台剧部酱~}
    \end{minipage}%
    }
    \hfill
    \adjustbox{valign=t}{
    \begin{minipage}[t]{0.65\textwidth}

        \normalsize
    \chind 大家好啊,这里是cosplay舞台剧部!\\
\chind 顾名思义,这里既有cosplay,又有舞台剧———是并集而不是交集。\\
\chind 部门的活动有许多,包括最盛大的社庆舞台剧节目、百团大战出cos、在动漫咖啡厅活动出cos、约漫展、约团片、约计划、约饭……\\
\chind 社庆的舞台剧节目上,我们有19年的命运石之门,21年的方舟走秀,24年的逆转裁判和25年的“苹果默示录”。\\
\chind 在百团大战中,社团会约好c楼的活动室方便大家化妆,一起在摊位上出cos也不社恐。在动漫主题咖啡厅中,我们还会设置符合主题的布景,让大家拍照使用。\\
\chind 大家可以组团去方舟ONLY、V家ONLY等漫展。\\

    
    \end{minipage}
    }
    \adjustbox{valign=t}{
    \begin{minipage}[t]{0.65\textwidth}

        \normalsize

\chind 约团片约计划什么的,只需要在群里说一声。万一成功了呢!这组JOJO黄金之风团片仅仅起源于一句话。并且结束以后一起去吃了披萨。\\
\chind 没出过cos,怎么办?尽管在群里提出问题吧!群内也有技术高超的妆娘、毛娘老师。试着在部门活动中迈出cosplay的第一步。\\
\chind 总之,欢迎所有对此有兴趣的同学加入!
\\  
    
    \end{minipage}
    }
    \hfill
  \adjustbox{valign=t}{
    \begin{minipage}[t]{0.25\textwidth}
      \vspace{-0.5em}
      \raisebox{-\height}{
      \includegraphics[width=\linewidth]{cos部3.jpg}}
      
      \picbox{\small ~~\ding{115} ~ JOJO黄金之风团片~}
    \end{minipage}%
    }
      \adjustbox{valign=t}{
    \begin{minipage}[t]{0.45\textwidth}
      \vspace{-2.5em}
      \raisebox{-\height}{
      \includegraphics[width=\linewidth]{cos部1.jpg}}
      
      \picbox{\small \ding{115} ~ 2024咖啡厅coser合影~}
    \end{minipage}%
    }  \adjustbox{valign=t}{
    \begin{minipage}[t]{0.45\textwidth}
      \vspace{0.5em}
      \raisebox{-\height}{
      \includegraphics[width=1.1\linewidth]{cos部2.jpg}}
      
      \picbox{\small ~~\ding{115} ~ 2024社庆舞台剧《逆转裁判》~}
    \end{minipage}%
    }
  

    \newpage
    \fontsize{23pt}{24pt}\selectfont
    \textbf{\textcolor{truepurple}{次世代心憩部}}\\
    \vspace{0.7em}
  \adjustbox{valign=t}{
    \begin{minipage}[t]{0.2\textwidth}
      \vspace{-0.2em}
      \raisebox{-\height}[0pt][0pt]{
      \includegraphics[width=1.1\linewidth]{心憩部.jpg}}
    \end{minipage}%
    }
    \hfill
    \adjustbox{valign=t}{
    \begin{minipage}[t]{0.7\textwidth}

        \small
\chind 欢迎加入心憩部/心理支援部捏!\\
\chind 本部建立的初衷是:互帮互助,让自己的生活更好一些——我们讨论如何改善身心问题与精神困扰,
也欢迎遇到困境的同学群友一起分享从看诊用药心得,到心理调节资源的各种知识,以及日常生活的体验与感想~\\
\chind 为什么会想在次世代动漫社里,建立一个以心理健康为主题的,听上去像抱团取暖(但其实并不是x)的社群呢?\\
\chind 加入次世代一段时间后,我始终忘不了一些社友在一些分部里面,倾诉自己遇到的学习生活困境
,回应他们的却或是无视或是委婉劝止(这里不是讨论沉重话题的地方),或是更多负面悲观的螺旋中。
这样做并不能让现实生活中的问题消失,并且就算有人提出了有价值的信息,也会旋即被水群的日常活动淹没乃至遗忘。\\

    
    \end{minipage}
    }
    \small

    \chind 我希望能真正帮助到大家,我也不认为逃避或者愤世嫉俗才是二次元的主基调,
成长和互帮互助一样是,因此有了这个试图分享心理健康资源,以及提供人文关怀和相互启发的心憩部,
或者也可以叫心理支援部(只是后者可能看起来像是走投无路的同学才会加入的样子,所以改成了前者)\\
\chind \textbf{爱是永不止息,无论在什么地方,以什么方式。}
\\  
\vspace{2em}    
\fontsize{23pt}{24pt}\selectfont
\textbf{\textcolor{truepurple}{次世代星露谷物语部}}\\
\vspace{0.2em}
\small
\chind 你说得对,但是星露谷物语是一个牧场类的RPG游戏,你继承了爷爷在星露谷的农场,
但是你手头上只有最基础的农具和少许金钱,你得靠此开始你的新生活。
你能把这片杂草丛生的田地变成繁荣的家园吗?\\
\chind 欢迎各位老乡加入,也欢迎感兴趣准备入手的新玩家捏\~{}
